\documentclass{article}
\usepackage{amsmath}
\usepackage{geometry}
\geometry{a4paper, margin=1in}

\title{IPv4/IPv6 Address and Subnet Calculation}
\author{Nikola Mihaylov}
\date{}

\begin{document}

\maketitle

\section*{IPv4}

\subsection*{Source Information}

Here is the source information to be used for the following subnetting process:

\begin{itemize}
    \item IP Address: \texttt{120.48.7.105} | \texttt{01111000.00110000.00000111.01101001}
    \item Subnet Mask: \texttt{255.255.255.248} | \texttt{11111111.11111111.11111111.11111000}
\end{itemize}

\subsection*{Network Address Calculation}
Calculating the network address involves performing an AND operation with the IP address and the subnet mask in binary:

\[
\begin{array}{c}
\texttt{01111000.00110000.00000111.01101001} \\
\text{AND} \\
\texttt{11111111.11111111.11111111.11111000} \\
\hline
\texttt{01111000.00110000.00000111.01101000} \\
\end{array}
\]

The resulting network address in decimal is:

\texttt{120.48.7.104}

\subsection*{Broadcast Address Calculation}
Calculating the broadcast address involves inverting the subnet mask and performing an OR operation with the network address.

Inverted subnet mask:

\[
\begin{array}{c}
\texttt{11111111.11111111.11111111.11111000} \\
\hline
\texttt{00000000.00000000.00000000.00000111} \\
\end{array}
\]

Performing the OR operation with the network address and the inverted subnet mask:

\[
\begin{array}{c}
\texttt{01111000.00110000.00000111.01101000} \\
\text{OR} \\
\texttt{00000000.00000000.00000000.00000111} \\
\hline
\texttt{01111000.00110000.00000111.01101111} \\
\end{array}
\]

The result converted from binary to decimal is:

\texttt{120.48.7.111}

\subsection*{First and Last Usable IP Addresses}

The first usable IP address is the network address plus one:

\[
\texttt{120.48.7.104} + 1 = \texttt{120.48.7.105}
\]

The last usable IP address is the broadcast address minus one:

\[
\texttt{120.48.7.111} - 1 = \texttt{120.48.7.110}
\]

\newpage

\section*{IPv6}

\subsection*{Source Information}

Here is the source information to be used for the following subnetting process:

\begin{itemize}
    \item IP Address: \texttt{2001:0db8:85a3:0000:0000:8a2e:0370:7334}
    \item Prefix Length: \texttt{/64}
\end{itemize}

\subsection*{Network Address Calculation}
Calculating the network address involves keeping the bits up to the prefix length and setting the remaining bits to zero:

\[
\begin{array}{c}
\texttt{2001:0db8:85a3:0000:0000:8a2e:0370:7334} \\
\text{AND} \\
\texttt{ffff:ffff:ffff:ffff:0000:0000:0000:0000} \\
\hline
\texttt{2001:0db8:85a3:0000:0000:0000:0000:0000} \\
\end{array}
\]

The resulting network address is:

\texttt{2001:db8:85a3::/64}

\subsection*{First and Last Usable IP Addresses}

The first usable IP address is the network address plus one. Since IPv6 addresses are hexadecimal, this operation is similar to adding one in binary:

\[
\texttt{2001:db8:85a3::1}
\]

The last usable IP address is the address before the next network address, calculated by setting all bits after the prefix to one and subtracting one:

\[
\begin{array}{c}
\texttt{2001:0db8:85a3:0000:0000:0000:0000:0000} \\
\text{OR} \\
\texttt{0000:0000:0000:ffff:ffff:ffff:ffff:ffff} \\
\hline
\texttt{2001:0db8:85a3:ffff:ffff:ffff:ffff:ffff} \\
\end{array}
\]

Subtracting one gives us:

\texttt{2001:db8:85a3:ffff:ffff:ffff:ffff:fffe}

\subsection*{Example Subnet Calculation}
Suppose we need to create four subnets within the \texttt{2001:db8:85a3::/64} network. This requires borrowing two bits from the host portion:

\begin{itemize}
    \item New Prefix Length: \texttt{/66}
\end{itemize}

The subnet addresses are then:

\begin{itemize}
    \item \texttt{2001:db8:85a3:0::/66}
    \item \texttt{2001:db8:85a3:0:0:0:0:4000::/66}
    \item \texttt{2001:db8:85a3:0:0:0:0:8000::/66}
    \item \texttt{2001:db8:85a3:0:0:0:0:c000::/66}
\end{itemize}

\end{document}
