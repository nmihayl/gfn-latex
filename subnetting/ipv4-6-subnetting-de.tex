\documentclass{article}
\usepackage{amsmath}
\usepackage{geometry}
\geometry{a4paper, margin=1in}

\title{IPv4/IPv6 Adress- und Subnetzberechnung}
\author{Nikola Mihaylov}
\date{}

\begin{document}

\maketitle

\section*{IPv4}

\subsection*{Quellinformationen}

Hier sind die Quellinformationen, die für den folgenden Subnetzprozess verwendet werden:

\begin{itemize}
    \item IP-Adresse: \texttt{120.48.7.105} | \texttt{01111000.00110000.00000111.01101001}
    \item Subnetzmaske: \texttt{255.255.255.248} | \texttt{11111111.11111111.11111111.11111000}
\end{itemize}

\subsection*{Berechnung der Netzwerkadresse}
Die Berechnung der Netzwerkadresse erfolgt durch eine UND-Operation mit der IP-Adresse und der Subnetzmaske in Binärform:

\[
\begin{array}{c}
\texttt{01111000.00110000.00000111.01101001} \\
\text{UND} \\
\texttt{11111111.11111111.11111111.11111000} \\
\hline
\texttt{01111000.00110000.00000111.01101000} \\
\end{array}
\]

Die resultierende Netzwerkadresse in Dezimalform ist:

\texttt{120.48.7.104}

\subsection*{Berechnung der Broadcast-Adresse}
Die Berechnung der Broadcast-Adresse erfolgt durch Invertieren der Subnetzmaske und eine ODER-Operation mit der Netzwerkadresse.

Invertierte Subnetzmaske:

\[
\begin{array}{c}
\texttt{11111111.11111111.11111111.11111000} \\
\hline
\texttt{00000000.00000000.00000000.00000111} \\
\end{array}
\]

Durchführen der ODER-Operation mit der Netzwerkadresse und der invertierten Subnetzmaske:

\[
\begin{array}{c}
\texttt{01111000.00110000.00000111.01101000} \\
\text{ODER} \\
\texttt{00000000.00000000.00000000.00000111} \\
\hline
\texttt{01111000.00110000.00000111.01101111} \\
\end{array}
\]

Das Ergebnis, konvertiert von Binär- in Dezimalform, ist:

\texttt{120.48.7.111}

\subsection*{Erste und letzte nutzbare IP-Adressen}

Die erste nutzbare IP-Adresse ist die Netzwerkadresse plus eins:

\[
\texttt{120.48.7.104} + 1 = \texttt{120.48.7.105}
\]

Die letzte nutzbare IP-Adresse ist die Broadcast-Adresse minus eins:

\[
\texttt{120.48.7.111} - 1 = \texttt{120.48.7.110}
\]

\newpage

\section*{IPv6}

\subsection*{Quellinformationen}

Hier sind die Quellinformationen, die für den folgenden Subnetzprozess verwendet werden:

\begin{itemize}
    \item IP-Adresse: \texttt{2001:0db8:85a3:0000:0000:8a2e:0370:7334}
    \item Präfixlänge: \texttt{/64}
\end{itemize}

\subsection*{Berechnung der Netzwerkadresse}
Die Berechnung der Netzwerkadresse erfolgt, indem die Bits bis zur Präfixlänge beibehalten und die verbleibenden Bits auf null gesetzt werden:

\[
\begin{array}{c}
\texttt{2001:0db8:85a3:0000:0000:8a2e:0370:7334} \\
\text{UND} \\
\texttt{ffff:ffff:ffff:ffff:0000:0000:0000:0000} \\
\hline
\texttt{2001:0db8:85a3:0000:0000:0000:0000:0000} \\
\end{array}
\]

Die resultierende Netzwerkadresse ist:

\texttt{2001:db8:85a3::/64}

\subsection*{Erste und letzte nutzbare IP-Adressen}

Die erste nutzbare IP-Adresse ist die Netzwerkadresse plus eins. Da IPv6-Adressen hexadezimal sind, ist dieser Vorgang dem Addieren von eins in Binärform ähnlich:

\[
\texttt{2001:db8:85a3::1}
\]

Die letzte nutzbare IP-Adresse ist die Adresse vor der nächsten Netzwerkadresse, berechnet durch Setzen aller Bits nach dem Präfix auf eins und Abzug von eins:

\[
\begin{array}{c}
\texttt{2001:0db8:85a3:0000:0000:0000:0000:0000} \\
\text{OR} \\
\texttt{0000:0000:0000:ffff:ffff:ffff:ffff:ffff} \\
\hline
\texttt{2001:0db8:85a3:ffff:ffff:ffff:ffff:ffff} \\
\end{array}
\]

Durch Abziehen von eins erhalten wir:

\texttt{2001:db8:85a3:ffff:ffff:ffff:ffff:fffe}

\subsection*{Beispiel Subnetzberechnung}
Angenommen, wir müssen vier Subnetze innerhalb des \texttt{2001:db8:85a3::/64}-Netzwerks erstellen. Dies erfordert das Ausleihen von zwei Bits aus dem Hostteil:

\begin{itemize}
    \item Neue Präfixlänge: \texttt{/66}
\end{itemize}

Die Subnetzadressen sind dann:

\begin{itemize}
    \item \texttt{2001:db8:85a3:0::/66}
    \item \texttt{2001:db8:85a3:0:0:0:0:4000::/66}
    \item \texttt{2001:db8:85a3:0:0:0:0:8000::/66}
    \item \texttt{2001:db8:85a3:0:0:0:0:c000::/66}
\end{itemize}

\end{document}
