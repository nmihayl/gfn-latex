\documentclass[]{article}

%opening
\title{Linux Commands}
\author{Nikola Mihaylov}

\begin{document}

\maketitle

\section{clear}

Clears the current terminal screen.

\section{pwd}

Prints the absolute path to the current working directory.

\section{--help}

Common flag that can be passed to the end of commands to display a short help page including options and the proper syntax of a command.

\section{man}

Manual page for a command. Equivalent of --help but much more detailed and scrollable.

\section{cd}

Change directory, accepts either a relative or absolute path.

\section{ls}

Alphabetically ists files and directories in the current working directory. By default it does not show hidden files and directories.

\section{su -}

Allows switching between users in current terminal session. Providing a hyphen to su switches to the root user account.

\section{whoami}

Prints the username of the current terminal session.

\section{history}

Prints a chronologically numbered list of the executed commands in the shell. Shortcut is Ctrl+r which opens a new search prompt for the hist file.

\section{!n}

Can be combined with history to repeat a command based on its chronological number in the history output.

\section{reboot}

Schedules a system reboot. Can be made instantaneous with the "now" option.

\section{shutdown}

Schedules a system shutdown. Can be made instantaneous with the "now" parameter. Also accepts the options "-r now" to instantly reboot and "-h 13:00" to schedule a shutdown using a 24 hour format clock argument.

\section{apt}

Package management utility for Debian-based distributions. Works as an abstraction for "dpkg" and can be substituted with "apt-get".

\section{ssh}

\section{systemctl}

\section{ip}
 
\end{document}
